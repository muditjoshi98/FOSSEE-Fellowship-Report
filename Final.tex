\documentclass[12pt,a4paper]{report}
\usepackage[hmargin=3cm,vmargin=3cm]{geometry}
\usepackage{graphicx}
\usepackage{wrapfig}
\usepackage{caption}
\usepackage{array}
\usepackage{listings}
\usepackage{color}
\usepackage{hyperref}
\hypersetup{
	colorlinks=true, % make the links colored
	linkcolor=blue, % color TOC links in blue
	urlcolor=cyan, % color URLs in red
	linktoc=all % 'all' will create links for everything in the TOC
}
\graphicspath{{images/}}

\begin{document}

\begin{figure}
\centering
\includegraphics[width = 0.3\textwidth]{iit}
\hspace{1cm}
\includegraphics[width = 0.4\textwidth]{fossee-logo.png}
\end{figure}

\title{\textbf{\textbf{Summer Fellowship Report}}\vspace{4mm} \\\small On \\\vspace{4mm} \textbf{\large Title}\\ \vspace{4mm}\small Submitted by\\  \vspace{4mm}  \large \textbf{Ashutosh Gangwar}\\ \vspace{1mm} B.Tech (Computer Science and Engineering)\\ MIET, Meerut\\ \vspace{4mm} \large \textbf{Mudit Joshi}\\ \vspace{1mm} B.Tech (Computer Science and Engineering)\\ PDPM IIITDM, Jabalpur\\ \vspace{4mm} \small Under the guidance of \\ \vspace{4mm}
	\large \textbf{Prof.Kannan M. Moudgalya} \vspace{1mm}\\ Chemical Engineering Department  \vspace{1mm} \\IIT Bombay
}

\maketitle


\newpage
\title{\textbf{\textbf{\LARGE 
\begin{flushleft}
\textbf{Acknowledgment}
\end{flushleft}
}}}
\begin{flushleft}
	The fellowship opportunity we had with FOSSEE Team was a great chance for learning and professional development. Therefore, we consider ourselves as very lucky as we were provided with an opportunity to be a part of it. We are also grateful for having a chance to meet so many wonderful people and professionals who led me though this internship period.
	
	Bearing in mind previous I am using this opportunity to express my deepest gratitude and special thanks to the MD of [Company name] who in spite of being extraordinarily busy with her/his duties, took time out to hear, guide and keep me on the correct path and allowing me to carry out my project at their esteemed organization and extending during the training.
	
	I express my deepest thanks to [Name Surname], [Position in the Company] for taking part in useful decision \& giving necessary advices and guidance and arranged all facilities to make life easier. I choose this moment to acknowledge his/her contribution gratefully.
	
	It is my radiant sentiment to place on record my best regards, deepest sense of gratitude to Mr./Ms. [Name Surname], [Position in the Company], Mr./Ms. [Name Surname], [Position in the Company], Mr./Ms. [Name Surname], [Position in the Company] and Mr./Ms. [Name Surname], [Position in the Company] for their careful and precious guidance which were extremely valuable for my study both theoretically and practically.
	
	I perceive as this opportunity as a big milestone in my career development. I will strive to use gained skills and knowledge in the best possible way, and I will continue to work on their improvement, in order to attain desired career objectives. Hope to continue cooperation with all of you in the future

\end{flushleft}

\listoffigures
\tableofcontents

\chapter{\textbf{Introduction}}
FOSSEE (Free and Open Source Software in Education) project promotes the use of FOSS tools to improve the quality of education in our country. They aim to reduce dependency on proprietary software in educational institutions. They encourage the use of FOSS tools through various activities to ensure commercial software is replaced by equivalent FOSS tools. They also develop new FOSS tools and upgrade existing tools to meet requirements in academia and research. 
Incorporated to FOSSEE program, this fellowship's main aim is to introduce students to the FOSS in various engineering fields and to become a part of this big community.
\\\\
We were selected for this fellowship on the basic of screen task submitted by us. There we got opportunity to work on some of the major open source electronic simulation softwares and are introduced to the Technology Stack they are build on. These technologies include C/C++ Programming, Python, Wxwidget, WxPython, PyQt4,etc. 
\\\\
At the beginning of the fellowship we formulated several learning goals, which we want to achieve:
\begin{itemize}
	\item To understand the functioning and working conditions of a government organisation
	\item To see what it is like to work in a professional environment
	\item To see if this kind of work is a possibility for our future career
	\item To use our knowledge and skills and to further increase them
	\item To learn about organising of a open source project
	\item To enhance our communication skills
	\item To build a professional and social network
\end{itemize}
This report is a short description of our 48 days fellowship under FOSSEE. This report contains our activities that have contributed to achieve a number of our stated goals. Following is the description of the softwares we worked on and changes we have done in them, concluding with the experience we gained.
\section{KiCad}
\href{http://kicad-pcb.org/}{KiCad} is a free software suite for electronic design automation (EDA). It facilitates the design of schematics for electronic circuits and their conversion to PCB designs. KiCad was originally developed by Jean-Pierre Charras. It features an integrated environment for schematic capture and PCB layout design. Tools exist within the package to create a bill of materials, artwork, Gerber files, and 3D views of the PCB and its components.
\\
The Kicad suit has 5 main parts: 
\begin{itemize}
	\itemsep0em 
	\item KiCad – the project manager.
	\item Eeschema – the schematic capture editor.
	\item Pcbnew – the PCB layout program. It also has a 3D view.
	\item GerbView – the Gerber viewer.
	\item Bitmap2Component – tool to convert images to footprints for PCB artwork.
\end{itemize}

\begin{figure}[h]
	\centering
	\includegraphics[scale=0.3]{KiCad-Logo}
	\caption{Kicad Logo}
	\vspace{5mm}
	\includegraphics[width=\textwidth]{KiCad}
	\caption{Kicad PcbNew OpenGL}
\end{figure}
\vspace{5mm}

\section{eSim}
\href{https://esim.fossee.in/}{eSim} (previously known as Oscad / FreeEDA) is an open source EDA tool for circuit design, simulation, analysis and PCB design. It is an integrated tool built using open source software such as \href{http://www.kicad-pcb.org}{KiCad} and \href{http://ngspice.sourceforge.net/}{NgSpice}.eSim is released under GPL.
\\
eSim offers similar capabilities and ease of use as any equivalent proprietary software for schematic creation, simulation and PCB design, without having to pay a huge amount of money to procure licenses. Hence it can be an affordable alternative to educational institutions and SMEs. It can serve as an alternative to commercially available/ licensed software tools like OrCAD, Xpedition and HSPICE.
The eSim suit Includes: 
\begin{itemize}
	\itemsep0em 
	\item KiCad  the complete KiCad suit.
	\item KiCadtoNgSpice - Generate Ngspice netlist.
	\item NgSpice Simulation - Simulate Circuit using NgSpice backend
	\item Model Editor
	\item Subcircuit Editor - Design subcircuit fro IC's 
	\item NGHDL - Convert VHDL to Ngspice
	\item Modelica Converter - Convert Modelica files to Schematic
	\item Modelica Optimization
\end{itemize}

\begin{figure}[h]
	\centering
	\includegraphics[scale=0.2]{eSim-logo}
	\caption{eSim Logo}
	\vspace{5mm}
	\includegraphics[width=11cm]{eSim}
	\caption{eSim Main Window}
\end{figure}


\section{ngSpice}
Ngspice is a mixed-level/mixed-signal circuit simulator. It is the open-source successor of Spice3f5. A small group of maintainers and the community of motivated users contribute to the ngspice project by providing new features, enhancements and bug fixes.

Ngspice is based on three free-software packages: Spice3f5, Xspice and Cider1b1:
\begin{itemize}
	\itemsep0em
	\item SPICE is the origin of all electronic circuit simulators, its successors are widely used in the electronics community.
	\item Xspice is an extension to Spice3 that provides additional C language code models to support analog behavioral modeling and co-simulation of digital components through a fast event-driven algorithm.
	\item Cider adds a numerical device simulator to ngspice. It couples the circuit-level simulator to the device simulator to provide enhanced simulation accuracy (at the expense of increased simulation time). Critical devices can be described with their technology parameters (numerical models), all others may use the original ngspice compact models.
\end{itemize}

Ngspice is, anyway, more than the simple sum of the packages above, as many people are contributing to the project with their experience, their bug fixes and their improvements giving ngspice additional features and improved robustness.

\begin{figure}[h]
	\centering
	\includegraphics[scale=0.5]{ngspice-logo}
	\caption{NgSpice Logo}
	\vspace{5mm}
	\includegraphics[height=7cm]{ngspice}
	\caption{ngSpice on KDE(Linux)}
\end{figure}


\chapter{\textbf{KiCad Nightly Build (v5)}}
\section{Building KiCad} 
\section{Bug : Autoplot PDF when saving projects}
This Bug : \verb!#! 1636549 is a feature addition in Kicad which is requested by a user of Kicad. It is somtime necesary to have the schematic in a handy format (like pdf) which can be easily accesed by other non-electric background people. So the requirement is to automatically plot the schematic in PDF format whenever the user saves the schematic, as the default plot method is time consuming.
\vspace{3mm}
\\
Bug Link : \url{https://bugs.launchpad.net/kicad/+bug/1636549}
\\
\\
Patch Link : \url{https://launchpadlibrarian.net/375310564/0001-Eeschema-Adding-Autoplot-PDF-when-saving-project.patch}
\begin{figure}[h]
	\centering
	\includegraphics[scale=0.4]{ki_bug_1}
	\caption{Required Mockup}
\end{figure}
\\
The schematic will be plot in PDF format and with scaling the color and the wire width will be taken from the previous setting keeping in mind that they are user defined property i.e. every user might have different requirement of PDF
\section{Bug : Add hotkey for opening context menu in eeschema}
This Bug : \verb!#! 1663595 is also a feature addition in eeschema in which the user wanted to have a shortcut to open the context menu for fast use of the software. The assigned hotkey for this function is \textbf{'D'}. The context menu open a bit below the cousor position.
\vspace{3mm}
\\
Bug Link : \url{https://bugs.launchpad.net/kicad/+bug/1663595}
\\
\\
Patch Link : \url{https://bugs.launchpad.net/kicad/+bug/1663595/+attachment/5159628/+files/0001-Eeschema-Add-shortcut-for-opening-context-menu.patch}
\begin{figure}[h]
	\centering
	\includegraphics[scale=0.3]{ki_bug_2}
	\caption{Patch Output}
\end{figure}
\\
\textbf{Solution} : Following is the Code Snippet to generate context menu
\begin{verbatim}
case HK_RIGHT_CLICK:
        {
            wxMenu MasterMenu;
            wxPoint pos1 = wxGetMousePosition() , pos2 = GetScreenPosition();
            wxPoint pos = pos1 - pos2;

            if( !OnRightClick( aPosition, &MasterMenu ) )
                return false;

            AddMenuZoomAndGrid( &MasterMenu );

            m_canvas->SetIgnoreMouseEvents(true);
            PopupMenu( &MasterMenu, pos );
            m_canvas->SetIgnoreMouseEvents(false);
        }
\end{verbatim}
\section{Bug : Ability to open project folder in host operating system}
\section{Bug : Inconsistent reference field parsing during editor copy}


\chapter{\textbf{Digital Simulation and Component Parser}}
This Chapter describes about the simulation in ngspice and about the component libraries of Kicad and eSim , similarities and differences between them. Problem description is to find out the reason behind the faliure of ngspice in simulating the digital circuits and some analog circuits in Kicad and to find out if we can use the components of kicad in eSim to increase the eSim library.
\section{Digital Simulation in KiCad}

\begin{wrapfigure}{r}{0.5\textwidth} %this figure will be at the right
	\centering
	\includegraphics[width=0.5\textwidth]{models}
	\caption{Models in ngSpice}
\end{wrapfigure}

Digital simulation is a problem coming in Kicad on simulating the circuits which works perfectly on eSim. Initially it was suggested that there must be some problem in the way Kicad converts the schematic to its respective generic netlist. On studying he kicad code base it is found that there is some changes in the way kicad identifies the connections from v4 (used in eSim) to v5 (new stable version of kicad). But it is not the only problem, because on changing the connection accordingly the same error pops i.e. \texttt{Error:model not found...} On refering to \href{http://ngspice.sourceforge.net/docs/ngspice-manual.pdf}{ngSpice manual}, it is found that there are some specific components which are identified as models in ngSpice which are shown.
\vspace{5mm}
\\
The problem found out is that for models other than those mention above ngSpice is not able to indetify them, and hence error. For such case ngSpice uses its festiure of subcircuit to make internal circuits for these components. Also among the shown components, basic components like R,C,L are automatically identified by ngSpice but for others i.e. Diodes and Transistors you have to specify them using \texttt{.model} function.
\vspace{5mm}
\\
In eSim, the Kicad to ngSpice Converter uses some of the hardcoded values for these models (stored in \texttt{.xml} format) specified in program and add these lines to the generic spice netlist. Also, in eSim for the models not specified in it, will fail to simulate. So the user have to make subcurcuits for these models in order to make them work.
\vspace{5mm}
\\
Even from the discussion in official \href{https://forum.kicad.info/t/digital-simulation/11072}{KiCad Forum}, it is suggested to make the subcircuits for the components, to make them work.

\section{Parser to increase supported components in eSim}
This section discuss about the suggestion to increase the no of components in eSim from Kicad by making a parser which convert the symbols from Kicad format to eSim format. It is suggested to take help from the \href{https://github.com/FOSSEE/Pspice-Kicad-Converter}{Pspice to Kicad parser} made earlier. On research it is found that the component files of Kicad and eSim are almost same. In fact eSim uses the eeschema software of kicad suit for schematic designing, so the components of kicad can be easily run on eSim, there might be only warning to uses a newer version of eeschema but it does not have any such effect in the running of the prorgam. And even the parser which was made earlier converts files from pspice to kicad which are two completely different softwares (infact competitors), whereas eSim is made on Kicad.
\\ Hence, it is conclude that there is no need to make a parser. In support to our conclusion , few components are added to eSim from Kicad which earlier dont work in eSim. These components include LM741, LM733H.

\subsection{LM741}
\begin{wrapfigure}{l}{0.4\textwidth} %this figure will be at the right
	\centering
	\includegraphics[width=0.4\textwidth]{lm741}
	\caption{LM741}
\end{wrapfigure}
The LM741 series are general-purpose operational amplifiers which feature improved performance over industry standards like the LM709. They are direct, plug-in replacements for the 709C, LM201, MC1439, and 748 in most applications.
\\
The amplifiers offer many features which make their application nearly foolproof: overload protection on the input and output, no latch-up when the common-mode range is exceeded, as well as freedom from oscillations.
\vspace{5mm}
\\
\begin{flushleft}
\textbf{Subcircuit file of LM741}
\end{flushleft}

\begin{verbatim}
* OPAMP MACRO MODEL (INTREMEDIATE LEVEL)
*                  IN- IN+ VEE    OUT  VCC
.SUBCKT lm741   18 2   1   102 19 81   101 20
Q1	5 1	7	NPN
Q2	6 2	8	NPN
RC1	101	5	95.49
RC2	101	6	95.49
RE1	7	4	43.79
RE2	8	4	43.79
I1	4	102	0.001
* OPEN-LOOP GAIN, FIRST POLE AND SLEW RATE
G1	100 10	6 5 0.0104719
RP1	10	100	9.549MEG
CP1	10	100	0.0016667UF
*OUTPUT STAGE
EOUT	80 100	10 100	1
RO	80	81	100
* INTERNAL REFERENCE
RREF1	101	103	100K
RREF2	103	102	100K
EREF	100 0	103 0 1
R100	100	0	1MEG
.model NPN  NPN(BF=50000)
.ENDS lm741
\end{verbatim}
\vspace{5mm}
\begin{figure}[h]
	\centering
	\includegraphics[scale=0.4]{lm741_sch}
	\caption{LM741 - Schematic}
\end{figure}

\vspace{5mm}
\begin{figure}[h]
	\centering
	\includegraphics[width=\textwidth]{lm741_sim}
	\caption{LM741 - Simulation Output}
\end{figure}

\subsection{LM733H}
\begin{wrapfigure}{l}{0.4\textwidth} %this figure will be at the right
	\centering
	\includegraphics[width=0.4\textwidth]{lm733h_sub}
	\caption{LM733H - Internal Subcircuit}
\end{wrapfigure}
The LM733/LM733C is a two-stage, differential input, differential output, wide-band video amplifier. The use of internal series-shunt feedback gives wide bandwidth with low phase distortion and high gain stability. Emitter-follower outputs provide a high current drive, low impedance capability. Its 120 MHz bandwidth and selectable gains of 10, 100 and 400, without need for frequency compensation, make it a very useful circuit for memory element drivers, pulse amplifiers, and wide band linear gain stages.
\vspace{5mm}
\\
\begin{flushleft}
	\textbf{Subcircuit file of LM733H}
\end{flushleft}

\begin{verbatim}
* Subcircuit LM733H
.subckt LM733H net-_q1-pad2_ net-_q1-pad3_ net-_r2-pad2_ net-_q3-pad3_ 
net-_r6-pad2_ net-_q3-pad2_ net-_r11-pad2_ net-_q10-pad1_ net-_q8-pad1_ 
net-_q10-pad3_ 

r2  net-_q1-pad3_ net-_r2-pad2_ 50
r6  net-_q3-pad3_ net-_r6-pad2_ 50
r3  net-_r2-pad2_ net-_q2-pad1_ 590
r7  net-_r6-pad2_ net-_q2-pad1_ 590
r4  net-_q2-pad3_ net-_r11-pad2_ 300
r9  net-_q4-pad3_ net-_r11-pad2_ 1.4k
r11  net-_q6-pad3_ net-_r11-pad2_ 300
r15  net-_q8-pad3_ net-_r11-pad2_ 400
r16  net-_q11-pad3_ net-_r11-pad2_ 400
r1  net-_q10-pad1_ net-_q1-pad1_ 2.4k
r5  net-_q10-pad1_ net-_q3-pad1_ 2.4k
r10  net-_q10-pad1_ net-_q10-pad2_ 1.1k
r14  net-_q10-pad1_ net-_q7-pad1_ 1.1k
r8  net-_q10-pad1_ net-_q11-pad2_ 10k
r12  net-_q8-pad1_ net-_q1-pad1_ 7k
r13  net-_q10-pad3_ net-_q3-pad1_ 7k
q1  net-_q1-pad1_ net-_q1-pad2_ net-_q1-pad3_ npn
q3  net-_q3-pad1_ net-_q3-pad2_ net-_q3-pad3_ npn
q7  net-_q7-pad1_ net-_q1-pad1_ net-_q5-pad3_ npn
q5  net-_q10-pad2_ net-_q3-pad1_ net-_q5-pad3_ npn
q9  net-_q10-pad1_ net-_q7-pad1_ net-_q8-pad1_ npn
q10  net-_q10-pad1_ net-_q10-pad2_ net-_q10-pad3_ npn
q11  net-_q10-pad3_ net-_q11-pad2_ net-_q11-pad3_ npn
q8  net-_q8-pad1_ net-_q11-pad2_ net-_q8-pad3_ npn
q6  net-_q5-pad3_ net-_q11-pad2_ net-_q6-pad3_ npn
q4  net-_q11-pad2_ net-_q11-pad2_ net-_q4-pad3_ npn
q2  net-_q2-pad1_ net-_q11-pad2_ net-_q2-pad3_ npn

.model npn  NPN( Is=14.34f Xti=3 Eg=1.11 Vaf=74.03 Bf=400 Ne=1.307 
Ise=14.34f Ikf=.2847 Xtb=1.5 Br=6.092 Nc=2 Isc=0 Ikr=0 Rc=1 Cjc=7.306p
Mjc=.3416 Vjc=.75 Fc=.5 Cje=22.01p Mje=.377 Vje=.75 Tr=46.91n Tf=411.1p
Itf=.6 Vtf=1.7 Xtf=3 Rb=10)

.ends LM733H
\end{verbatim}
\vspace{5mm}
\begin{figure}[h]
	\centering
	\includegraphics[scale=0.3]{lm733h_sch}
	\caption{LM733H - Schematic}
\end{figure}

\vspace{5mm}
\begin{figure}[h]
	\centering
	\includegraphics[width=\textwidth]{lm733h_sim}
	\caption{LM733H - Simulation Output}
\end{figure}
\vspace{5mm}
\textbf{\Large Conclusion}
\vspace{5mm}
\\
So after all research and discussion it is concluded that the solution for both digital simulation and parser problem is to make the subcircuit for every components required in the circuit. Also, Kicad being a PCB designing focused software give more preference to the shape of IC and the positioning of pins in the IC rather than what is the internal circuitory of the component. Also the above subcircuits for LM741 and LM733H support the solution. But the problem which arised in this case is the limited no of ports to represent pins of the subcircuit (which is only 8), whose solution is discussed in the next chapter.



\chapter{\textbf{eSim}}

\section{Increase External Pins for Sub-Circuits}
\section{Introduced Rename Project Option}
\section{Improve handling of unknown components}
\section{Introduced workspace functionality in eSim}
\section{Minor Bug Fix and improving code usability}

\chapter{\textbf{Standalone Installer for eSim}}

\chapter{\textbf{Conclusion and future work}}

\chapter{\textbf{References}}

\end{document}


